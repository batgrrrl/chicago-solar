The \ac{uhi} effect, where urban areas tend to be warmer than their surroundings,
is one of the most evident impacts of human activity. Indeed, \ac{uhi} is one of
the most well studied phenomena, often using remote sensing technology and land
surface temperature data \cite{almeida_study_2021, cotlier_extreme_2022}.
Some studies evaluated \ac{uhi} intensity for particular cities by incorporating
data about urban features such as albedo, building height, and vegetation
\cite{sangiorgio_development_2020,abulibdeh_analysis_2021} along with urbanization
trends \cite{li_how_2021}. Other work in the literature examined \ac{uhi} along
a socio-economic axis.

\begin{itemize}
  \item What papers have looked at uhi and class?
  \item What papers have looked at uhi mitigation strategies?
  \item What are the most frequent mitigation strategies (green+white roofs)
  \item What papers have looked at the solar panel distribution?
  \item What are the social aspects of solar panel distribution?
  \item How have solar panels been distributed previously? (Introduce CEJST & EJSCREEN)
  \item What gap in the literature is this work specifically filling?
\end{itemize}




Various studies have addressed heat wave problem and distribution of solar panels separately till now. An assessment by means of Land Surface Temperature using remote sensing technology \cite{cotlier_extreme_2022}, the mapping of heat stress by crowdsourcing geospatial data and high spatial resolution data and evaluation of socioeconomic characteristics \cite{maragno_mapping_2020}, quantifying synergies between Urban Heat Island effect and heatwaves in urban areas \cite{founda_synergies_2017}  are some of the studies addressing heatwave problem. While these studies mainly discuss about the spatial distribution of heat stress, others have also addressed its implications to the society by assessing socio-economic vulnerability and risk. One of the study\cite{maragno_mapping_2020} has developed sensitivity, adaptive capacity, vulnerability, exposure, and risk indicators and tested the framework on urban area concluding that vulnerability and risk levels differ with changing locational characteristics within the same urban area and hence the high resolution spatial dataset serves a great importance to plan necessary adaptation strategies. This study served as a foreground to choose the high resolution datasets in our study. Our study area is Chicago city which is an highly urbanized area with varying locational characteristics. We reviewed few studies related to Chicago heatwaves. The study by Bernice Ackerman  has addressed the effect of Lake Michigan on temperature of its surrounding area and the effect of green cover. It was helpful in identifying the crucial indicators like green cover and presence of water bodies which help surrounding areas to stay relatively cooler and hence could serve as good adaptive capacity indicators.

Prior research has shown works that study the distributional disparities in residential rooftop solar potential in four major cities in Chicago and a few other cities \cite{reames_distributional_2020}. The study found that the highest rooftop potential in Chicago was in census tracts with higher percentages of \ac{lmi} households. The \ac{lmi} households represented 51\% of solar suitable households in Chicago. However, the lower penetration of solar \ac{lmi} communities substantially decreases the overall attainment of renewable energy and energy equity goals in Chicago.  DeepSolar, a machine learning framework that efficiently constructs a solar deployment dataset for the United States has found that the solar deployment density is strongly correlated and decreases with the Gini index, a measure of income inequality \cite{yu_deepsolar_2018}. This points out how socio-economic inequality causes disparities in solar distribution. Most states across the United States have developed one or more policies to incentivize distributed solar PV investments. Many states have adopted various additional financial incentives to encourage and support the deployment of customer-owned distributed solar energy systems \cite{pitt_assessing_2015}. These policies and incentives are similar to Illinois Shine and Solar for All programs in Illinois which the current study focuses on.
However, it has also been shown that the distribution of low-carbon technology subsidies and their associated benefits can be highly uneven across socioeconomic groups, revealing a persistent inequality issue. The high income community usually have the resources and knowledge required to avail the benefits of such subsidies and hence tend to be more benefited than their low income counterparts \cite{stewart_all_2021}. This escalates the need for equitable distribution of the solar incentives across different socioeconomic communities. Hence, this study aims to find the ‘high risk’ areas in Chicago and help facilitate equitable distribution of solar panels through the Illinois Shine and Solar for all programs.
